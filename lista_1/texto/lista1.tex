% !TeX spellcheck = pt_BR
%%%%%%%%%%%%%%%%%%%%%%%%%%%%%%%%%%%%%%%%%
% Programming/Coding Assignment
% LaTeX Template
%
% This template has been downloaded from:
% http://www.latextemplates.com
%
% Original author:
% Ted Pavlic (http://www.tedpavlic.com)
%
% Note:
% The \lipsum[#] commands throughout this template generate dummy text
% to fill the template out. These commands should all be removed when 
% writing assignment content.
%
% This template uses a Perl script as an example snippet of code, most other
% languages are also usable. Configure them in the "CODE INCLUSION 
% CONFIGURATION" section.
%
%%%%%%%%%%%%%%%%%%%%%%%%%%%%%%%%%%%%%%%%%

%----------------------------------------------------------------------------------------
%	PACKAGES AND OTHER DOCUMENT CONFIGURATIONS
%----------------------------------------------------------------------------------------

\documentclass{article}

\usepackage{fancyhdr} % Required for custom headers
\usepackage{lastpage} % Required to determine the last page for the footer
\usepackage{extramarks} % Required for headers and footers
\usepackage[usenames,dvipsnames]{color} % Required for custom colors
\usepackage{graphicx} % Required to insert images
\usepackage{listings} % Required for insertion of code
\usepackage{courier} % Required for the courier font
\usepackage{lipsum} % Used for inserting dummy 'Lorem ipsum' text into the template
\usepackage[brazil]{babel}
\usepackage[utf8]{inputenc}
\usepackage[T1]{fontenc}
\usepackage{amsmath}
\usepackage{amssymb}
\usepackage{epstopdf}
\usepackage{subcaption}
\usepackage{myMacros}
\usepackage{listings}

% Margins
\topmargin=-0.45in
\evensidemargin=0in
\oddsidemargin=0in
\textwidth=6.5in
\textheight=9.0in
\headsep=0.25in

\linespread{1.1} % Line spacing

% Set up the header and footer
\pagestyle{fancy}
\lhead{\hmwkAuthorShortName} % Top left header
\chead{\hmwkTitle} % Top center head
\rhead{\firstxmark} % Top right header
\lfoot{\lastxmark} % Bottom left footer
\cfoot{} % Bottom center footer
\rfoot{Pág.\ \thepage\ de\ \protect\pageref{LastPage}} % Bottom right footer
\renewcommand\headrulewidth{0.4pt} % Size of the header rule
\renewcommand\footrulewidth{0.4pt} % Size of the footer rule

\setlength\parindent{0pt} % Removes all indentation from paragraphs

%----------------------------------------------------------------------------------------
%	CODE INCLUSION CONFIGURATION
%----------------------------------------------------------------------------------------

\definecolor{MyDarkGreen}{rgb}{0.0,0.4,0.0} % This is the color used for comments
\lstloadlanguages{Matlab} % Load Perl syntax for listings, for a list of other languages supported see: ftp://ftp.tex.ac.uk/tex-archive/macros/latex/contrib/listings/listings.pdf

\lstset{language=Matlab, % Use Perl in this example
        frame=single, % Single frame around code
        basicstyle=\small\ttfamily, % Use small true type font
        keywordstyle=[1]\color{Blue}\bf, % Perl functions bold and blue
        keywordstyle=[2]\color{Purple}, % Perl function arguments purple
        keywordstyle=[3]\color{Blue}\underbar, % Custom functions underlined and blue
        identifierstyle=, % Nothing special about identifiers                                         
        commentstyle=\usefont{T1}{pcr}{m}{sl}\color{MyDarkGreen}\small, % Comments small dark green courier font
        stringstyle=\color{Purple}, % Strings are purple
        showstringspaces=false, % Don't put marks in string spaces
        tabsize=5, % 5 spaces per tab
        %
        %FIX para o problema de enconding
        extendedchars=true,
        literate={á}{{\'a}}1 {ã}{{\~a}}1 {é}{{\'e}}1 {ê}{{\^e}}1 {í}{{\'i}}1 {ú}{{\'u}}1 {ó}{{\'o}}1 {õ}{{\~o}}1 {ç}{{\c{c}}}1,
        % Put standard Perl functions not included in the default language here
        morekeywords={rand, exprnd,rng},
        %
        % Put Perl function parameters here
        morekeywords=[2]{on, off, interp},
        %
        % Put user defined functions here
        morekeywords=[3]{test},
       	%
        morecomment=[l][\color{Blue}]{...}, % Line continuation (...) like blue comment
        numbers=left, % Line numbers on left
        firstnumber=1, % Line numbers start with line 1
        numberstyle=\tiny\color{Blue}, % Line numbers are blue and small
        stepnumber=5 % Line numbers go in steps of 5
}

%Muda Listing #: para Trecho #
\renewcommand{\lstlistingname}{Trecho}

% Creates a new command to include a perl script, the first parameter is the filename of the script (without .pl), the second parameter is the caption
\newcommand{\script}[2]{
\begin{itemize}
\item[]\lstinputlisting[caption=#2,label=#1]{#1}
\end{itemize}
}

%----------------------------------------------------------------------------------------
%	DOCUMENT STRUCTURE COMMANDS
%	Skip this unless you know what you're doing
%----------------------------------------------------------------------------------------

% Header and footer for when a page split occurs within a problem environment
\newcommand{\enterProblemHeader}[1]{
\nobreak\extramarks{#1}{#1 continua na próxima página\ldots}\nobreak
\nobreak\extramarks{#1 (continuação)}{#1 continua na próxima página\ldots}\nobreak

}

% Header and footer for when a page split occurs between problem environments
\newcommand{\exitProblemHeader}[1]{
\nobreak\extramarks{#1 (continuação)}{#1 continua na próxima página\ldots}\nobreak

\addtocounter{homeworkProblemCounter}{1}
\nobreak\extramarks{Questão \arabic{homeworkProblemCounter}}{}\nobreak
\addtocounter{homeworkProblemCounter}{-1}
}

\setcounter{secnumdepth}{0} % Removes default section numbers
\newcounter{homeworkProblemCounter} % Creates a counter to keep track of the number of problems

\newcommand{\homeworkProblemName}{}


\newenvironment{homeworkProblem}[1][Questão \arabic{homeworkProblemCounter}]
{ % Makes a new environment called homeworkProblem which takes 1 argument (custom name) but the default is "Problem #"
\stepcounter{homeworkProblemCounter} % Increase counter for number of problems
\setcounter{homeworkItemCounter}{0}
\renewcommand{\homeworkProblemName}{#1} % n \homeworkProblemName the name of the problem
\section{\homeworkProblemName} % Make a section in the document with the custom problem count
\enterProblemHeader{\homeworkProblemName} % Header and footer within the environment
}
{
\exitProblemHeader{\homeworkProblemName} % Header and footer after the environment
}

%MODIFICAÇÕES ROBERTO
\newcounter{homeworkItemCounter}
%\newcommand{\homeworkItemName}{}
%\newenvironment{homeworkItem}[1][Item \alph{homeworkItemCounter}]{
%\stepcounter{homeworkItemCounter}
%\renewcommand{\homeworkProblemName}{#1}
%\subsection{\homeworkItemName}
%}


\newcommand{\problemAnswer}[1]{ % Defines the problem answer command with the content as the only argument
\noindent\framebox[\columnwidth][c]{\begin{minipage}{0.98\columnwidth}#1\end{minipage}} % Makes the box around the problem answer and puts the content inside
}

\newcommand{\homeworkSectionName}{}
\newenvironment{homeworkSection}[1][Item \alph{homeworkItemCounter})]{ % New environment for sections within homework problems, takes 1 argument - the name of the section
\stepcounter{homeworkItemCounter}
\renewcommand{\homeworkSectionName}{#1} % Assign \homeworkSectionName to the name of the section from the environment argument
\subsection{\homeworkSectionName} % Make a subsection with the custom name of the subsection
\enterProblemHeader{\homeworkProblemName\ [\homeworkSectionName]} % Header and footer within the environment
}{
\enterProblemHeader{\homeworkProblemName} % Header and footer after the environment
}

%----------------------------------------------------------------------------------------
%	NAME AND CLASS SECTION
%----------------------------------------------------------------------------------------

\newcommand{\hmwkTitle}{Lista\ de\ Exercícios\ \#1} % Assignment title
\newcommand{\hmwkDueDate}{Quinta,\ 21\ de\ Março,\ 2019} % Due date
\newcommand{\hmwkClass}{CPE723 Otimização Natural} % Course/class
\newcommand{\hmwkClassTime}{Terças e Quintas: 08:00-10:00} % Class/lecture time
\newcommand{\hmwkClassInstructor}{Prof. José Gabriel Rodríguez Carneiro Gomes} % Teacher/lecturer
\newcommand{\hmwkAuthorName}{Vinicius Mesquita de Pinho} % Your name
\newcommand{\hmwkAuthorShortName}{Vinicius Mesquita} % Nome que aparece no cabeçalho

%----------------------------------------------------------------------------------------
%	TITLE PAGE
%----------------------------------------------------------------------------------------

\title{
\vspace{2in}
\textmd{\textbf{\hmwkClass:\\ \hmwkTitle}}\\
\normalsize\vspace{0.1in}\small{Data\ de\ entrega:\ \hmwkDueDate}\\
\vspace{0.1in}\large{\textit{\hmwkClassInstructor,\ \hmwkClassTime}}
\vspace{3in}
}

\author{\textbf{\hmwkAuthorName}}
\date{} % Insert date here if you want it to appear below your name

%----------------------------------------------------------------------------------------

\DeclareMathAlphabet{\mathpzc}{OT1}{pzc}{m}{it}
\newcommand{\z}{\mathpzc{z}}
\renewcommand{\v}[1]{\boldsymbol{\mathbf{#1}}}

\begin{document}

\maketitle

%----------------------------------------------------------------------------------------
%	TABLE OF CONTENTS
%----------------------------------------------------------------------------------------

%\setcounter{tocdepth}{1} % Uncomment this line if you don't want subsections listed in the ToC

\clearpage
\newpage
%\tableofcontents
%\newpage

%----------------------------------------------------------------------------------------
%	PROBLEM 1
%----------------------------------------------------------------------------------------

% To have just one problem per page, simply put a \clearpage after each problem

\setcounter{homeworkProblemCounter}{0}
\begin{homeworkProblem}
	Calcular $\int_{0}^{1} xe^{-x}dx$ de três formas diferentes.

	\begin{homeworkSection}[a) Primeiro calculando a integral indefinida.]
		
		\begin{equation}
		\int x e^{-x} dx = -e^{-x}x - e^{-x} + k,
		\end{equation}
		onde k é uma constante, que daqui para frente assumiremos $k = 0$. Aplicando integral por parte com $u = x$ e $v^{\prime} = e^{-x}$, teremos
		\begin{equation}
		-e^{-x}x - e^{-x} + k = -e^{-x} - \int - e^{-x} dx.
		\end{equation}
		Como $\int - e^{-x}dx = e^{-x}$, teremos
		\begin{equation}
		-e^{-x} - \int - e^{-x} dx = -e^{-x}x - e^{-x}.
		\end{equation}
		Calculando os limites:
		\begin{equation}
		\int_{0}^{1}xe^{-x}dx = -\frac{2}{e} - (-1) = -\frac{2}{e} + 1 \approx 0.26424\dots 
 		\end{equation}
 		
	\end{homeworkSection}

	\begin{homeworkSection}[b) Pelo método de Monte Carlo, usando 10 números escolhidos aleatoriamente com densidade uniforme entre 0 e 1.]
		
	Os números gerados: $ \Ical =  ( 0.66606563, 0.1280393 , 0.15019057, 0.53518197, 0.72382377,
	0.56306944, \\ 0.18701318, 0.96037606, 0.79777229, 0.23135354)$
	
	O cálculo:
	\begin{equation}
		\label{eq:integral_aproximada}
		\int_{0}^{1} xe^{-x}d \approx \dfrac{\sum_{i \in \Ical} ie^{-i}}{|\Ical|} 
	\end{equation}
		onde $|\Ical|$ é a cardinalidade do conjunto $\Ical$.
	
	Para o conjunto $\Ical$ o resultado foi $0.2634$.
	\end{homeworkSection}
	\begin{homeworkSection}[c) Pelo método de Monte Carlo, usando 10 números escolhidos aleatoriamente com densidade exponencial (note que as amostras geradas a partir da p.d.f. exponencial devem ser limitadas ao intervalo de 0 a 1).]
		
		Os números gerados: $ \Wcal =  (0.55005713, 0.39015194, 0.23349916, 0.41284879, 0.22265732, \\ 0.69167714, 0.43932533, 0.09646257, 0.60023859, 0.98892791)$
		
		Utilizando a equação~\eqref{eq:integral_aproximada} para o conjunto $\Wcal$, o resultado encontrado foi $0.2632$.

	\end{homeworkSection}
\end{homeworkProblem}
\clearpage
%----------------------------------------------------------------------------------------
\newpage

\begin{homeworkProblem}
	Usando $ N = 20 $ números aleatórios, escolhidos a partir de uma p.d.f. uniforme entre $-1$ e $ +1 $, calcular uma aproximação para o número $\pi$ pelo método de Monte Carlo. Faça o mesmo no computador, utilizando um valor alto para N (por exemplo, 1.000.000). Comente o resultado.



A fórmula para $\pi$ utilizada é,
\begin{equation} 
\label{eq:pi}
\pi = \int_{-1}^{1} \frac{dx}{\sqrt{1-x^{2}}}
\end{equation}

Primeiro, para o $N= 20$. 

Os números gerados: $ \Qcal =  (-0.85289404,  0.92237068,  0.32455622,  0.0946911 ,  0.8676302 ,
-0.92568284,  0.76944606, \\ -0.90241725,  0.20247619, -0.69989244,
-0.64724326,  0.91118243,  0.2354838 ,  0.8327723 , \\ -0.21647745,
-0.5894811 , -0.81595565, -0.61292112,  0.12647561, -0.33242752)$

O cálculo:
\begin{equation}
\label{eq:pi_aproximado}
\int_{-1}^{1} \frac{dx}{\sqrt{1-x^{2}}} \approxeq \dfrac{\sum_{q \in \Qcal} \left(\sqrt{1-q^{2}}\right)^{-1}}{|\Qcal|} 
\end{equation}
onde $|\Qcal|$ é a cardinalidade do conjunto $\Qcal$.

Utilizando o conjunto $\Qcal$, o resultado obtido foi $3.142843$.

Para um valor de $N$ alto, foi utilizado $N = 10^{7}$, onde o valor obtido foi $3.1457290$.

A lei dos grandes números garante que a com um número maior amostras utilizadas, a aproximação converge para o resultado da integral.


\end{homeworkProblem}
\clearpage
\newpage
%----------------------------------------------------------------------------------------

\begin{homeworkProblem}
	
	Escrever um algoritmo para gerar números $x(n)$ com energia $J(x) = x^{2}$, de forma que as probabilidades dos números gerados ejam proporcionais aos fatores de Boltzmann $e^{-J(x)/T}$, com temperatura $T = 0.1$. Começando de um valor $x(0)$ qualquer, aplique sempre perturbações $\epsilon R$ ao valor $x(n)$ atual. Neste caso, R é uma vriável aleatória uniforme. Considere $\epsilon = 0.1$.
	
	\begin{homeworkSection}[a) Execute o algoritmo proposto no computador, calculando $x(n)$ até $n = 100.000$.]
	O algoritmo em Python.
		\begin{lstlisting}[language=Python]
		# Ponto inicial x = 0
		x_inicial = np.array(0)
		
		J_inicial = funcao_J(x_inicial)
		x_atual = x_inicial
		J_atual = J_inicial
		
		# Parametros utilizados
		N = 10**5
		T_inicial = 0.1
		T = T_inicial
		epsilon = 0.1
		
		fim = 0
		n = 0
		k = 1
		J_min = J_atual
		x_min = x_atual
		
		# Para armazenar os J e os x
		todos_J = np.array([]), todos_x = np.array([]
		
		while not(fim):
			n = n + 1
			x = x_atual + epsilon*(np.random.uniform(-1, 1))
			J = funcao_J(x)
			todos_J = np.append(todos_J,J)
			todos_x = np.append(todos_x,x)
			if (np.random.uniform(0,1) < np.exp((J_atual-J)/T)):
				x_atual = x
				J_atual = J
			if (J < J_min):
				J_min = J
				x_min = x
			if (n % N == 0):
				fim = 1
		\end{lstlisting}
		
		
	\end{homeworkSection}
	\newpage
	\begin{homeworkSection}[b) Execute manualmente os 10 primeiros passos do algoritmo.]
		
	Utilizando a mesma lógica proposta na letra a), executamos os 10 primeiros passos do algoritmo, manualmente.
	
% Please add the following required packages to your document preamble:
% \usepackage{graphicx}
\begin{table}[!h]
	\centering
	\resizebox{\textwidth}{!}{%
		\begin{tabular}{c|c|c|c|c|c|c}
			\multicolumn{1}{l|}{Iteração} & x atual & x candidato & J(x candidato) & r (uniforme) & q                               & r \textless q \\ \hline
			\textbf{1}                    & 0.0000  & 0.0703      & 0.8262         & 0.9646       &  $\sim$ 3300 & V             \\
			\textbf{2}                    & 0.0703  & 0.0770      & 0.7146         & 0.8961       & 3.0537                          & V             \\
			\textbf{3}                    & 0.0770  & 0.1033      & 0.3512         & 0.2745       & 37.8699                         & V             \\
			\textbf{4}                    & 0.1033  & 0.1109      & 0.2666         & 0.3008       & 2.3294                          & V             \\
			\textbf{5}                    & 0.1109  & 0.2090      & -0.2062        & 0.1587       & 113.0382                        & V             \\
			\textbf{6}                    & 0.2090  & 0.2666      & -0.1403        & 0.0232       & 0.5177                          & V             \\
			\textbf{7}                    & 0.2666  & 0.2908      & -0.0771        & 0.0128       & 0.5312                          & V             \\
			\textbf{8}                    & 0.2908  & 0.3590      & 0.1326         & 0.8441       & 0.1229                          & F             \\
			\textbf{9}                    & 0.2908  & 0.3750      & 0.1782         & 0.5509       & 0.0779                          & F             \\
			\textbf{10}                   & 0.2908  & 0.3051      & -0.0345        & 0.5210       & 0.6532                          & V            
		\end{tabular}%
	}
\end{table}
	
	
	\end{homeworkSection}


\end{homeworkProblem}
\clearpage
\newpage

\begin{homeworkProblem}
Escrever um programa de S.A. (pode ser pseudo-código) para minimizar a função escalar $J(x) = -x + 100(x - 0.2)^{2}(x-0.8)^{2}$. Começando de $x(0) = 0$ e utilizando geradores de números aleatórios (um uniforme e outro gaussiano), calcule manualmente os 10 primeiros valores de $x(n)$ gerados pelo S.A.


\begin{lstlisting}[language=Python]
# Definindo a funcao
def funcao_J(x):
	return -x + 100*((x-0.2)**2)*((x-0.8)**2)

# Definindo ponto inicial
x_inicial = np.array(0)

J_inicial = funcao_J(x_inicial)
x_atual = x_inicial
J_atual = J_inicial

# Parametros utilizados
N = 1000 
K = 8
T_inicial = 5e-3
T = T_inicial
epsilon = 10e-2

fim = 0
n = 0
k = 1
J_min = J_atual
x_min = x_atual

todos_J = np.array([])
todos_x = np.array([])

while not(fim):
	n = n + 1
	x = x_atual + epsilon*(np.random.normal(0, 1))
	J = funcao_J(x)
	todos_J = np.append(todos_J,J)
	todos_x = np.append(todos_x,x)
	if (np.random.uniform(0,1) < np.exp((J_atual-J)/T)):
		x_atual = x
		J_atual = J
	if (J < J_min):
		J_min = J
		x_min = x
	if (n % N == 0):
		k = k + 1
		T = T_inicial/(np.log(1+k))
	if k == K:
		fim = 1


\end{lstlisting}

\newpage
O cálculo manual dos dez primeiros valores de x(n):

Neste caso, quando o valor de  q $= \exp((J \, atual-J)/T)$ é muito grande, representei por $\sim$inf na tabela.

\begin{table}[!h]
	\centering
	\resizebox{\textwidth}{!}{%
		\begin{tabular}{c|c|c|c|c|c|c}
			\multicolumn{1}{l|}{Iteração} & x atual & x candidato & J(x candidato) & r $\sim$ U(0,1) & q         & r \textless q \\ \hline
			\textbf{1}                    & 0.0000  & 0.0775      & 0.7051         & 0.0940       & $\sim$inf & V             \\
			\textbf{2}                    & 0.0775  & -0.0023     & 2.6372         & 0.3660       & 0         & F             \\
			\textbf{3}                    & 0.0775  & 0.0766      & 0.7205         & 0.6785       & 0.0454    & F             \\
			\textbf{4}                    & 0.0775  & 0.1179      & 0.1953         & 0.2147       & $\sim$inf & V             \\
			\textbf{5}                    & 0.1179  & 0.1338      & 0.0604         & 0.8711       & $\sim$inf & V             \\
			\textbf{6}                    & 0.1338  & 0.1121      & 0.2538         & 0.7671       & 0         & F             \\
			\textbf{7}                    & 0.1338  & 0.1154      & 0.2204         & 0.9551       & 0         & F             \\
			\textbf{8}                    & 0.1338  & 0.2042      & -0.2036        & 0.5003       & $\sim$inf & V             \\
			\textbf{9}                    & 0.2042  & 0.1346      & 0.0545         & 0.1564       & 0         & F             \\
			\textbf{10}                   & 0.2042  & 0.4453      & 0.3118         & 0.5321       & 0         & F            
		\end{tabular}%
	}
\end{table}


\end{homeworkProblem}
%----------------------------------------------------------------------------------------
\clearpage
\newpage
\begin{homeworkProblem}
	Proponha uma função de até 4 variáveis cujo ponto mínimo você conheça, e encontre este ponto mínimo utilizando S.A. (neste exercício, basta entregar o código escrito).
	
	A função escolhida:
	\begin{equation}
	f(x,y) = (x-3)^{2} + xy + (y-1)^{2}
	\end{equation}
	Essa função tem mínimo em $\{x,y\} = \{\frac{10}{3},-\frac{2}{3}\}$.


	O código em Python escrito para solução do problema:
	
	\begin{lstlisting}[language=Python]
	
# Definindo a funcao (que recebe um pont pt que eh funcao de x e y)
def chosen_function(pt):
	x = pt[0]
	y = pt[1]
	return (x-3)**2 + x*y + (y-1)**2

# Ponto inicial	
pt_inicial = np.array([0,0])

J_inicial = chosen_function(pt_inicial)
pt_atual = pt_inicial
J_atual = J_inicial

# Parametros utilizados
N = 1000 
K = 8
T_inicial = 5e-1
T = T_inicial
epsilon = 10e-2

fim = 0
n = 0
k = 1
J_min = J_atual
pt_min = pt_atual

while not(fim):
	n = n + 1
	pt = pt_atual + epsilon*(np.random.normal(0, 1, 2))
	J = chosen_function(pt)
	if (np.random.uniform(0,1) < np.exp((J_atual-J)/T)):
		pt_atual = pt
		J_atual = J
	if (J < J_min):
		J_min = J
		pt_min = pt
	if (n % N == 0):
		k = k + 1
		T = T_inicial/(np.log(1+k))
	if k == K:
		fim = 1
	
	
	
	\end{lstlisting}
		
\end{homeworkProblem}


\end{document}