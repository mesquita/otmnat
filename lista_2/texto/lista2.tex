% !TeX spellcheck = pt_BR

%%%%%%%%%%%%%%%%%%%%%%%%%%%%%%%%%%%%%%%%%%%%%%%
% Modelo adaptado do template original de
% Ted Pavlic (http://www.tedpavlic.com)
% Todos os créditos a ele.
%
% Na versão atual, o que foi modificado
% do original:
% Ajusta a numeração das questões e
% passa para português.
% Além de separar as configurações
% em um arquivo .cls separado.
%
% Crédito ao Roberto por ter feito
% a maior parte do trabalho de passar
% para o português e fazer outros
% ajustes para a versão atual deste template.
%%%%%%%%%%%%%%%%%%%%%%%%%%%%%%%%%%%%%%%%%%%%%%%


%----------------------------------------------------------------------------------------
%	PACKAGES E OUTRAS CONFIGURAÇÕES
%----------------------------------------------------------------------------------------

\documentclass{homeworkclass}

\usepackage{myMacros}


\hmwkTitle{Lista\ de\ Exercícios \#2}
\hmwkDueDate{Quinta,\ 28\ de\ Março,\ 2019}
\hmwkClass{CPE723 Otimização Natural}
\hmwkClassTime{Terças e Quintas: 08:00-10:00}
\hmwkClassInstructor{Prof. José Gabriel Rodríguez Carneiro Gomese}
\hmwkAuthorName{Vinicius Mesquita de Pinho}
\hmwkAuthorShortName{Vinicius Mesquita}

\begin{document}

\maketitle

%----------------------------------------------------------------------------------------
%	SUMÁRIO
%----------------------------------------------------------------------------------------

%\setcounter{tocdepth}{1} % Uncomment this line if you don't want subsections listed in the ToC

\clearpage
\newpage
%\tableofcontents
%\newpage

%----------------------------------------------------------------------------------------
%	QUESTÃO 1
%----------------------------------------------------------------------------------------

% To have just one problem per page, simply put a \clearpage after each problem


\begin{homeworkProblem}

Considere um processo de Markov $X(t)$ que tem três estados possíveis: 0, 1 e 2. A evolução temporal deste processo é dada pela matriz de transição a seguir:
\begin{equation*}
M = \begin{bmatrix}
0.5 & 0.25 & 0.25 \\
0.25 & 0.50 & 0.25 \\
0.25 & 0.25 & 0.50
\end{bmatrix}
\end{equation*}
\begin{homeworkSection}[a) Considerando que a distribuição de probabilidade de $X(0)$ é dada pelo vetor $\pbf_{0} = \left( 0.3 \, \, 0.4 \, \, 0.3 \right)^{\Trm}$, calcule a distribuição de probabilidade de $X(3)$ (ou seja, do processo de Markov no instante $t = 3$).]

A distribuição de probabilidade de $X(1)$ pode ser calculada a partir da cadeia de Markov da seguinte maneira:
\begin{equation*}
	\pbf_{1} = \Mbf \pbf_{0}.
\end{equation*}
A distribuição de probabilidade de $X(2)$, seguindo a mesma ideia:
\begin{equation*}
	\pbf_{2} = \Mbf^{2} \pbf_{1}.
\end{equation*}
Logo, a distribuição de probabilidade de $X(3)$, pode ser obtida por:
\begin{equation*}
	\pbf_{3} = \Mbf^{3} \pbf_{2}.
\end{equation*}
Usando os valores dados pela questão, obtemos o seguinte vetor $\pbf_{3}^{\Trm} = \left( 0.3593 \, \, 0.3489 \, \, 0.3463 \right)$.
\end{homeworkSection}


\begin{homeworkSection}[b)Iniciando em $X(0) = 1$, e usando um gerador de números aleatórios (são necessários apenas três números aleatórios equiprováveis), calcule manualmente uma amostra do processo X(t) até $t = 3$.]

O código usado para a questão.
\begin{lstlisting}[language=Python]
	M = np.matrix('0.50 0.25 0.25; 0.25 0.50 0.25; 0.25 0.25 0.50')
	estado_atual = 1
	tempos_possiveis = np.array([1, 2, 3])
	soma_coluna = 0
	iterador = 0

	for t in tempos_possiveis:
		uniforme_sorteado = np.random.uniform(0,1)
		while uniforme_sorteado > soma_coluna:
			soma_coluna = soma_coluna + M[estado_atual,iterador]
			iterador += 1
		print(f'no tempo {t} => mudança de estado: de {estado_atual} para 
		... {iterador}. uniforme sorteado foi {uniforme_sorteado:.3}')
		estado_atual = iterador	
\end{lstlisting}

As saídas encontradas: \\
No tempo 1 => Mudança de estado: de 1 para 2. Uniforme sorteado foi 0.441. \\
No tempo 2 => Mudança de estado: de 2 para 2. Uniforme sorteado foi 0.564. \\
No tempo 3 => Mudança de estado: de 2 para 3. Uniforme sorteado foi 0.773.
\end{homeworkSection}


\begin{homeworkSection}[c) Usando um computador, execute 100 repetições do item (b). Em cada uma das 100 repetições, comece a simulação com um valor diferente de $X(0)$, assumindo que os eventos $X(0) = 0$, $X(0) = 1$ e $X(0) = 2$ são equiprováveis. Armazene as 100 cadeiras obtidas em uma matriz $\Xbf$, com 4 colunas ($t = 0$ até $t = 3$) e 100 linhas.]
\end{homeworkSection}

\begin{homeworkSection}[d) Fazendo histogramas de cada uma das 4 colunas, calcule as distribuições de probabilidade do processo $X(t)$ em cada um dos 4 instantes: $t = 0, \, 1 \, 2 \, 3$. Comente os resultados obtidos.]


\end{homeworkSection}




\end{homeworkProblem}
\clearpage
%----------------------------------------------------------------------------------------
%	QUESTÃO 2
%----------------------------------------------------------------------------------------

\begin{homeworkProblem}
Considere um sistema em que só há 5 estados possíveis: $x = 1$, $x = 2$, $x = 3$, $x = 4$ e $x = 5$. Os custos de $J(x)$ De cada um dos estados são indicados na tabela abaixo:

\begin{center}
	\begin{tabular}{|c|c|}
		\hline
		$x$ & $J(x)$ \\
		\hline
		1 & 0.5 \\
		\hline
		2 & 0.2 \\
		\hline
		3 & 0.3 \\
		\hline
		4 & 0.1 \\
		\hline
		5 & 0.5 \\
		\hline
	\end{tabular}
\end{center}

\begin{homeworkSection}[a) Considere um processo de Markov gerado pela aplicação do algoritmo de Metropolis aos dados da tabela acima, com temperatura fixa $T = 0.1$. Calcule a matriz de transição $M$ que define o processo de $X(t)$. Obs: note que o estado $X(t)$ é unidimensional, e portanto a matriz $M$ é $5 \times 5$.]
\end{homeworkSection}

\begin{homeworkSection}[b) Iniciando em $X(0) = 1$, calcule manualmente 4 amostras do processo X(t).]
\end{homeworkSection}

\begin{homeworkSection}[c) Qual é o vetor invariante da matriz $M$ do item (a)? Obs: para facilitar os cálculos, pode-se usar o computador neste item.]
\end{homeworkSection}

\begin{homeworkSection}[d) Calcule os fatores de Boltzmann (ou seja, $e^{-(J(x)/T)}$) associados aos dados da tabela acima, e compare-os com o resultado do item (c). Use $T = 0.1$.]
\end{homeworkSection}

\begin{homeworkSection}[e) Simulated Annealing: Usando um computador, execute 1000 iterações do algoritmo de Metropolis em cada uma das 10 temperaturas a seguir. Na passagem de uma temperatura para a outra, use o estado atual. Comente as distribuições de probabilidade obtidas no final de cada temperatura.]

\end{homeworkSection}



\end{homeworkProblem}
%----------------------------------------------------------------------------------------
%	QUESTÃO 3
%----------------------------------------------------------------------------------------

\begin{homeworkProblem}
Proponha uma função $J(\xbf)$, sendo $\xbf$ um vetor com 10 dimensões, cujo ponto mínimo você conheça. Evite propor funções que tenham um só ponto mínimo. Encontre o ponto mínimo global utilizando S.A. Obs: Neste exercício, entregue o código utilizando e alguns comentários sobre o resultado obtido.
\end{homeworkProblem}


\end{document}